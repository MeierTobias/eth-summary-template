% Define all the needed variable like it is shown in the var.tex.example
% (Just copy the var.tex.example file and remove the ".example" from the filename.)
% specify the relative path from the main file to the folder that contains the template repo
\newcommand{\templatePath}{..}

% define some information about this summary
\newcommand{\summaryTitle}{Summary Title}
\newcommand{\summarySubTitle}{HS23 ETHZ}
\newcommand{\summaryAuthor}{Juri Pfammatter, Daniel Schweizer and Tobias Meier}
\newcommand{\repoURL}{https://github.com/MeierTobias/eth-summary-template}
\newcommand{\summaryInfo}{This PDF, the source code as well as the disclaimer can be found in the GitHub repository \url{\repoURL}.}
\newcommand{\imagePath}{./images/}

% layout options
\newcommand{\orientationmode}{landscape} % landscape or portrait
\newcommand{\papersize}{a4paper} % a4paper or a3paper
\newcommand{\fontheight}{10pt}

% additional conditions
\newcommand{\includeexamples}{1} % 0 = don't show examples, 1 = show examples

\documentclass[\fontheight]{extarticle}

% load the template headers

% document setup
\usepackage[left=7mm,right=7mm,top=20mm,bottom=0mm,landscape]{geometry} % margin = .. total={280mm,190 mm} % in geometry for defined size/ratio
\usepackage{multicol,multirow}
\usepackage[utf8]{inputenc} % not strictly necessary, but sets utf8

% enable colors
\usepackage{xcolor,color} % standard colors (blue, red, etc.https://www.namsu.de/Extra/pakete/Xcolor.html )

% multicol settings
\setlength{\premulticols}{1pt}
\setlength{\postmulticols}{1pt}
\setlength{\multicolsep}{1pt}
\setlength{\columnsep}{5pt}
\setlength{\columnseprule}{1pt}
\def\columnseprulecolor{\color{black}}

% language
\usepackage[english]{babel} %choose your language

% for images
\usepackage{graphicx}
\graphicspath{ {\imagePath{}} }

% Images combined with texts
\usepackage{wrapfig}

% some AsmTeX options
\usepackage{amscd, amsmath,amssymb}

% more fine control for lists
\usepackage{enumitem}
\setlist{noitemsep}
%\setlist{nosep}

% multiple line comments and testing text
\usepackage{comment} % \begin{comment} \end{comment} 
\usepackage{blindtext} % inserts lorem ipsum like text

% table that supports equations
\usepackage{tabularx}
\usepackage{booktabs}

% define header style
\usepackage[headsepline]{scrlayer-scrpage}
\pagestyle{scrheadings}
\ihead{\summaryTitle{} - \today}
\chead{\pagemark}
\ohead{\url{\repoURL}}
\setlength{\headsep}{5pt}

% hyperlinks have to be included last
\usepackage{hyperref}
\hypersetup{
    colorlinks,
    citecolor=blue,
    filecolor=black,
    linkcolor=black,
    urlcolor=black,
    pdftitle={\summaryTitle{}}
}

% set the section numbering down to the paragraph level
\setcounter{secnumdepth}{4}
\setcounter{tocdepth}{4}

% column separation
\providecommand{\newcol}{\vfill\null\columnbreak}
% colors
\usepackage{color,xcolor}
\definecolor{sectionColor}{HTML}{15836c}  % 1c59c9
\definecolor{subSectionColor}{HTML}{1c59c9} % 568ae8
\definecolor{subSubSectionColor}{HTML}{568ae8} % 87a9e8
\definecolor{paragraphColor}{HTML}{87a9e8} % 87a9e8
\definecolor{subParagraphColor}{HTML}{87a9e8} % 87a9e8
\definecolor{exampleColor}{HTML}{c3c2c5}
\definecolor{titleTextColor}{RGB}{255,255,255}

% multicol settings
\setlength{\premulticols}{1pt}
\setlength{\postmulticols}{1pt}
\setlength{\multicolsep}{1pt}
\setlength{\columnsep}{5pt}
\setlength{\columnseprule}{1pt}
\def\columnseprulecolor{\color{lightgray}}

% define header style
\usepackage[headsepline]{scrlayer-scrpage}
\pagestyle{scrheadings}
\ihead{\summaryTitle{} - \today}
\chead{\pagemark}
\ohead{\url{\repoURL}}
\setlength{\headsep}{5pt}

% hyperlinks have to be included last
\usepackage{hyperref}
\hypersetup{
  colorlinks,
  citecolor=blue,
  filecolor=black,
  linkcolor=black,
  urlcolor=black,
  pdftitle={\summaryTitle{}}
}

% set the size of a section
\usepackage{parskip}
\setlength{\parindent}{0pt}
\setlength{\parskip}{0pt}

% used for colored boxes
\usepackage[many]{tcolorbox}

% customize the headers
\usepackage[explicit]{titlesec}
\usepackage{titletoc}

% creates a custom title background 
\newcommand{\customtitlebackground}[4]{\begin{tcolorbox}[
      enhanced,
      boxrule=0pt,
      arc=0pt,
      outer arc=0pt,
      left=0pt,
      right=0pt,
      top=0pt,
      bottom=0pt,
      nobeforeafter,
      interior code={\fill[overlay,#1] (frame.north west) rectangle (frame.south east);},
    ]\textcolor{#2}{#3\hspace{0.5em}#4}
  \end{tcolorbox}}
  
% Update section title (same for all section levels)
% This creates a title with colored box at the background
% The default colors are specified at the top of this file
% The new \section command also accepts an optional color parameter to use a custom background color.
% \section[customColor]{Section name}
% More in detail:
% The middle part (\titleformat and \titlespacing) defines the layout. The \backgroundcolor command
% is used as a placeholder for later. The renewcommand at the bottom adds the functionality to
% accept an optional color parameter.
\newcommand\backgroundcolor{}
\titleformat{\section}
{\normalfont\bfseries\fontfamily{lmss}\selectfont}{}{0pt}
{\customtitlebackground{\backgroundcolor}{titleTextColor}{\thesection}{#1}}[{\startcontents[section]}]
\titlespacing{\section}{0pt}{1pt}{0pt}
% set background color as an optional parameter
\let\oldsection\section
\renewcommand\section[2][sectionColor]{\edef\backgroundcolor{#1}\oldsection{#2}}

% update subsection title
\titleformat{\subsection}
{\normalfont\bfseries\fontfamily{lmss}\selectfont}{}{0pt}
{\customtitlebackground{\backgroundcolor}{titleTextColor}{\thesubsection}{#1}}
\titlespacing{\subsection}{0pt}{1pt}{0pt}
% set background color as an optional parameter
\let\oldsubsection\subsection
\renewcommand\subsection[2][subSectionColor]{\edef\backgroundcolor{#1}\oldsubsection{#2}}

% update subsubsection title
\titleformat{\subsubsection}
{\normalfont\bfseries\fontfamily{lmss}\selectfont}{}{0pt}
{\customtitlebackground{\backgroundcolor}{titleTextColor}{\thesubsubsection}{#1}}
\titlespacing{\subsubsection}{0pt}{1pt}{0pt}
% set background color as an optional parameter
\let\oldsubsubsection\subsubsection
\renewcommand\subsubsection[2][subSubSectionColor]{\edef\backgroundcolor{#1}\oldsubsubsection{#2}}

% update paragraph title
\titleformat{\paragraph}
{\normalfont\bfseries\fontfamily{lmss}\selectfont}{}{0pt}
{\customtitlebackground{paragraphColor}{titleTextColor}{\theparagraph}{#1}}
\titlespacing{\paragraph}{0pt}{1pt}{0pt}
% set background color as an optional parameter
\let\oldparagraph\paragraph
\renewcommand\paragraph[2][paragraphColor]{\edef\backgroundcolor{#1}\oldparagraph{#2}}

% update subparagraph title
\titleformat{\subparagraph}
{\normalfont\bfseries\fontfamily{lmss}\selectfont}{}{0pt}
{\customtitlebackground{subParagraphColor}{titleTextColor}{\thesubparagraph}{#1}}
\titlespacing{\subparagraph}{0pt}{1pt}{0pt}
% set background color as an optional parameter
\let\oldsubparagraph\subparagraph
\renewcommand\subparagraph[2][subParagraphColor]{\edef\backgroundcolor{#1}\oldsubparagraph{#2}}

% Create a TOC for the current section
\newcommand{\createsectiontoc}{
  \printcontents[section]{p}{2}{}{}%
}
% Removes the page numbering from the sub- and subsub-section
\titlecontents{subsection}[0.2em]{}{\thecontentslabel\hspace{1em}}{}{}[]
\titlecontents{subsubsection}[2.6em]{}{\thecontentslabel\hspace{1em}}{}{}[]
\titlecontents{paragraph}[5.7em]{}{\thecontentslabel\hspace{1em}}{}{}[]

% Example section environment
\newenvironment{examplesection}[1][]
{
  \ifnum\value{section}=0
    \PackageError{examplesection}{Can't place an examplesection here}{Create a section first.}
  \else
    \ifnum\value{subsection}=0
      \subsection{#1}
    \else
      \ifnum\value{subsubsection}=0
        \subsubsection{#1}
      \else
        \ifnum\value{paragraph}=0
          \paragraph{#1}
        \else
          \subparagraph{#1}
        \fi
      \fi
    \fi
  \fi

  \color{blue}
}
{
}

% exclude examples if specified
\if 0\includeexamples
  \excludecomment{examplesection}
\fi
%Mathematik-Pakete
\usepackage{amsmath, amstext, amssymb, mathtools, esint, polynom, trfsigns, pgfplots}
%Spec. version of pgfplots (rel. new version 2023)
\pgfplotsset{compat=1.18}
\usetikzlibrary{backgrounds}
\usepackage{bm}
%Seitenumbruch in align-Umgebung erlauben
\allowdisplaybreaks{}
%

%Definition der Umgebung "example"
\newenvironment{example}
{\begin{itshape} \begin{small}}
			{\end{small} \end{itshape}}
%				
%Definition der Umgebung "annotation"		
\newenvironment{annotation}[1]
{\begin{itshape} \begin{small} \textbf{#1} \begin{itemize}}
				{\end{itemize} \end{small} \end{itshape}}
%				
%Definition der Umgebung "eq"
\newenvironment{eq}
{\begin{equation*}}
		{\end{equation*}}
%
% Don't know what this does
\providecommand{\diff}{\mathop{} \! \mathrm{d}}
\DeclareMathOperator{\rot}{rot}
\DeclareMathOperator{\divg}{div}

% define colors
\definecolor{mathGreen}{RGB}{0, 120, 40}
 % ChkTex 27

\begin{document}
\begin{multicols*}{3}

    % create the title section
    
\begin{center}
    \null
    \huge{\summaryTitle{}\vspace*{0.4cm}\par}
    \large{\summarySubTitle{}\vspace*{0.2cm}\par}
    \normalsize{\summaryAuthor{}\vspace*{0.3cm}}
\end{center}

\summaryInfo{}
 % ChkTex 27

    % ---------------------------------------------------------------
    %                     your content goes here
    % ---------------------------------------------------------------

    % These are some example sections
    \section{Section}
    You can structure your summary with the following five section types.
    \begin{itemize}
        \item \texttt{\symbol{92}section}
        \item \texttt{\symbol{92}subsection}
        \item \texttt{\symbol{92}subsubsection}
        \item \texttt{\symbol{92}paragraph}
        \item \texttt{\symbol{92}ptitle}
    \end{itemize}

    \subsection{Sub-Section}
    This is a sub-section.

    \subsubsection{Sub-Sub-Section}
    This is a sub-sub-section.

    \paragraph{Paragraph}
    This is a paragraph.

    \ptitle{Paragraph Title}
    And this is a paragraph title.

    \section{Math examples}

    Use either \texttt{equation}, \texttt{align} or \texttt{gather} to
    state mathematical formulations.

    \subsection{Single equation}
    The \textit{D'Alembert} solution for a 1D Wave-PDE is given by:
    \begin{equation*}
        u(x,t)=\frac{1}{2}\left[f(x+ct)+f(x-ct)\right]+\frac{1}{2c}\int_{x-ct}^{x+ct}g(s)ds
    \end{equation*}

    \subsection{Multiline equation}
    Simple multiline equation can be written with the \texttt{gather} environment.

    \begin{gather*}
        u(x,t) =\sum_{n=1}^\infty B_n\sin(\frac{n\pi}Lx)e^{-\lambda_n^2 A(t)} \\
        A(t)=\int a(s)\;ds
    \end{gather*}

    To get aligned equations sue the \texttt{align} environment.
    \begin{align*}
        u(t-a)       \; & \laplace\; \frac{1}{s}e^{-as} \\
        f(t-a)u(t-a) \; & \laplace\; e^{-as}F(s)
    \end{align*}

    \textbf{Hint:} \LaTeX{} is very strict about equations at the beginning of a section.
    If you add an \texttt{align} environment at the start of an section use the
    \symbol{92}\texttt{noindent} command before the equation to suppress the unnecessary
    line break.

    \section{Section TOC}
    This template provides a command to generate a section specific table of content.
    \createsectiontoc{}

    \subsection{First level of mini TOC}
    \subsubsection{Second level of mini TOC}
    \paragraph{Third level of mini TOC}

    \newcol{}

    \section{Section in next column}
    This section is in a new column.

    \section{Examples}
    To add an example you can use the \texttt{examplesection} environment.
    \begin{examplesection}[Example title]
        Here is your example.
    \end{examplesection}

    \section{Code}
    \textbf{Inline code formatting} can be achieved by using \code{\symbol{92}code\{\}} for regular sized code
    and \code{\symbol{92}fncode\{\}} for footnote sized code.\newline
    Regular size: \code{this is some inline code},\newline
    Footnote sized: \fncode{this is smaller code}.
    \newpar{}
    To achieve \textbf{formatted code blocks}, use \newline
    \code{\symbol{92}begin\{lstlisting\}[style=bright\_C++]\newline
    Code Block\newline
    \symbol{92}end\{lstlisting\}}

    \begin{lstlisting}[style=bright_C++]
#include <algorithm>
std::vector<int> unsorted;

struct File{
    unsigned handle;
}
std::sort(unsorted.begin(),unsorted.end());
std::sort(  unsorted_custom.begin(),
            unsorted_custom.end(),
            [](auto fp1, auto fp2){
                return fp1->handle <fp2->handle;
            });
\end{lstlisting}

    Currently, custom code coloring is only implemented for C++, for python etc.\ use the default style.
    \newpar{}
    \textbf{Remark} inside the listing, the absolute number of tabs is used.

    \section{Lists and Enumerations}
    \begin{itemize}
        \item Lists can be used to list things
              \begin{enumerate}
                  \item first
                  \item second
              \end{enumerate}
        \item Change indent with option \code{[leftmargin = 40pt]}
              \begin{enumerate}[leftmargin=40pt]
                  \item enumerate with larger indent
              \end{enumerate}
    \end{itemize}

    \section{Table}
    To create aesthetically pleasing tables, \code{tabularx} and \code{booktabs} can be used:
    \newpar{}
    \renewcommand{\arraystretch}{1.3}
    \setlength\tabcolsep{6pt} % default value: 6pt
    \begin{tabularx}{\linewidth}{@{}llcp{0.4\linewidth}@{}}
        \toprule
                                                                & \multicolumn{3}{c}{Multicolumn}                                                                                        \\
        \cmidrule{2-4}
        \multirow{2}{*}{\begin{sideways}Multirow\end{sideways}} & left aligned                    & centered & paragraph with automatic linebreak    \\
        \cmidrule{2-4}
        \morecmidrules\cmidrule{2-4}
        &                                 &          & custom new\newline lines                                                                         \\
        \bottomrule
    \end{tabularx}
    \renewcommand{\arraystretch}{1}
    \setlength\tabcolsep{6pt} % default value: 6pt
    \textbf{Remark:} When using table, try to reduce the number of horizontal lines and avoid vertical lines at all.
    Most of the time the alignment of columns and rows is more than sufficient.

    \section{A Remark on Spacing}
    To make this template more general, the spacing between objects is held to a minimum and spaces can be inserted manually when needed.
    \code{\symbol{92}newpar\{\}} can be used to add a vertical space of 4pt.


    \definecolor{customSectionColor}{HTML}{990000}
    \section[][customSectionColor]{Custom section}
    This is how you can use a custom section color.

    \section[Alternative title text Sigma]{Alternative title text $\Sigma$}
    This is how you specify a alternative title text.

    \tableofcontents

    % ---------------------------------------------------------------
    %                     your content ends here
    % ---------------------------------------------------------------

\end{multicols*}
\end{document}